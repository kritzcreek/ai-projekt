\chapter{Fazit}
\label{cha:fazit}
Die Entwicklung der Anwendung FROST war ein Erfolg. Ich konnte alle gewünschten
Funktionalitäten wie in~\hyperref[cha:die-anwendung]{Kapitel 2} beschrieben
umsetzen. Die Applikation läuft nun seit mehreren Monaten fehlerfrei und stabil.
Mithilfe der Administrator Ansicht lassen sich neue Timetables anlegen, ohne
dass die Applikation neu gestartet werden muss.

Diese Stabilität und Korrektheit ist zu großen Teilen auf die in
Kapitel~\hyperref[cha:architektur]{Kapitel 3} beschriebene Architektur
zurückzuführen. Die Wahl der Programmiersprache Haskell hat sich hier sowohl für
die Effizienz als auch die Korrektheit der Software ausgezahlt.

Die Entscheidung für WebSockets im Zusammenspiel mit Event Sourcing lässt die
Anwendung dem Nutzer schnelles Feedback geben und sorgt für eine angenehme
Nutzererfahrung.

Die in~\hyperref[cha:user-interface]{Kapitel 4} beschriebenen Designziele konnte
umgesetzt werden. Die Anwendung erlaubt eine intuitive Nutzung ohne
Verzögerungen. Im Bezug auf die Usability habe ich von bisherigen Nutzern nur
positives Feedback erhalten.

Die Verwendung von PureScript~(\hyperref[cha:purescript]{Kapitel 5}) hat mir großen Spaß gemacht. Das ausdrucksstarke
Typsystem hat mir bei der Entwicklung von FROST immer wieder geholfen und eine
hohe Entwicklungsgeschwindigkeit ermöglicht. Weiterhin habe ich in der
PureScript Community eine tolle Quelle für neue Ideen und Anregungen gefunden.


%%% Local Variables:
%%% mode: latex
%%% TeX-master: "../ai-projekt"
%%% End: