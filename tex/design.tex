\chapter{User Interface \& User Experience}
\label{cha:user-interface}
In diesem Kapitel werde ich auf Design Entscheidungen im Bezug auf das User
Interface und die Interaktionen des Nutzers mit der Anwendung eingehen.\\
Hierbei habe ich einige Entscheidungen getroffen, die ich im folgendem
aufgreifen und einzeln begründen werde.

\section{Keine Authentifizierung/Authorisierung}
Das Ziel ist/war es die Anwendung so spontan und offen wie möglich zu gestalten.
Ein Loginvorgang würde diesem Prinzip im Weg stehen, wobei eine Authorisierung
für den Administrationsbereich eventuell nötig ist wenn die Größe der
Teilnehmerzahl über einen Punkt wächst, ab dem sich einige User zu anonym fühlen
und Schabernack treiben wollen.
Authentifizierung im Sinne einer Wiedererkennung desselben Users könnte
Usability Vorteile bringen und ist langfristig geplant. Das wichtigste Kriterium
für das gewählte Verfahren ist aber auf jeden Fall die Offenheit und Spontanheit
der Anwendung.

\section{Drag n' Drop}
Ich habe mich für Drag n' Drop als primäres Interaktionsverfahren in FROST
entschieden. Drag n' Drop ist ein intuitives Verfahren um mit den in FROST
vorhandenen Elementen zu interagieren. Das Konzept eines Stundenplans, auf dem
man Themenblöcke hin- und herschieben kann, ist leicht auf die Wirklichkeit
abzubilden. Weiterhin ist Drag n' Drop ein besonders geeignetes Verfahren auf
Touchgeräten, bei dem der Mensch ``physikalisch'' mit den Benutzerelementen
interagiert.

\section{Visuelles Design}
FROST bemüht sich keinen visuellen Overhead für den Nutzer zu erzeugen. Das Ziel
ist eine minimale, ansprechende Nutzeroberfläche, bei der man nichts entfernen
kann ohne wesentliche Funktionalität einzuschränken. So soll sich der Nutzer
möglichst schnell zurecht finden können.

%%% Local Variables:
%%% mode: latex
%%% TeX-master: "../ai-projekt"
%%% End:
