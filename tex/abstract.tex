\begin{abstract}
Funktionale Programmierung gewinnt aufgrund ihrer Vorteile im Umfeld von
Parallelisierung und Skalierung immer mehr an Bedeutung. Diese Vorteile zeigen
sich primär in Backend-Systemen, wo große Cluster von Servern, Daten verarbeiten.

Besonders bekannt ist in diesem Umfeld Haskell, eine lazy evaluierte
Programmiersprache mit einem mächtigen statischem Typsystem. Die
Garantien, die ein solches Typsystem mit sich bringt, ermöglichen es auch große
Codebases langfristig zu warten und zu erweitern. Auch der Compiler macht sich
diese Garantien zu Nutze und kann großflächig optimieren.

Doch auch am Frontend gilt es mittlerweile komplexe Systeme zu entwickeln.
Besonders im Bereich der Webentwicklung werden die Nutzeroberflächen schöner,
schneller und vor allem \emph{komplexer}. Leider führen Browser nur JavaScript,
eine Scriptsprache mit vielen Schwächen und Kuriositäten aus, die den
Herausforderungen, die interaktive Nutzeroberflächen mit sich bringen, nicht
gewachsen ist.

An dieser Stelle setzt PureScript an, eine rein funktionale Programmiersprache
die zu JavaScript kompiliert wird. Mit einem ausdrucksstarken Typsystem und den
nötigen Werkzeugen für Abstraktionen wie Monaden und Higher-Order-Polymorphismen
erlaubt es die Möglichkeiten funktionaler Programmierung für das Bändigen der
Komplexität am Frontend zu nutzen.

Im Zuge meiner Ausbildung habe ich ein Projekt mit Haskell und PureScript
umgesetzt und werde in dieser Arbeit von meinen Erfahrungen und Erkenntnissen,
sowie der entstandenen Applikation FROST, berichten.

\end{abstract}

%%% Local Variables:
%%% mode: latex
%%% TeX-master: "../ai-projekt"
%%% End:
